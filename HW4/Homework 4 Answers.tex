\documentclass[a4paper]{article}

%% Language and font encodings
\usepackage[english]{babel}
\usepackage[utf8x]{inputenc}
\usepackage[T1]{fontenc}

\usepackage{listings}
\usepackage{color} %red, green, blue, yellow, cyan, magenta, black, white
\definecolor{mygreen}{RGB}{28,172,0} % color values Red, Green, Blue
\definecolor{mylilas}{RGB}{170,55,241}


%% Sets page size and margins
\usepackage[a4paper,top=3cm,bottom=2cm,left=3cm,right=3cm,marginparwidth=1.75cm]{geometry}
\pagenumbering{gobble}

%% Useful packages
\usepackage{amsmath}
\usepackage{enumerate}
\usepackage{enumitem}
\setlist[enumerate,1]{start=1}
\setlength\parindent{0pt}
\usepackage{amssymb}
\usepackage{graphicx}
\usepackage{float}
\usepackage{caption}

\title{MA 595 Homework 4}

\begin{document}
\maketitle

\section*{Practice Problems}

\subsection*{Reference}

$$
w_1 = e^{2\pi i} = 1,\quad w_2 = e^{\pi i} = -1, \quad w_3 = e^{2\pi i / 3},\quad w_4 = e^{\pi i /2 } = i, \quad w_n = e^{2 \pi i / n}
$$

$$
F_1 = [1], \quad F_2 = \begin{bmatrix} 1&1 \\ 1&-1\end{bmatrix},
\quad F_4 =
\begin{bmatrix}
1 & 1 & 1 & 1 \\
1 & i &  -1 & -i \\
1 & -1 & 1 & -1 \\
1 & -i & -1 & i
\end{bmatrix},
$$
$$
(F_n)_{jk} = (w_n)^{jk} = e^{2\pi i j k /n} \quad j,k \in \{0,1,\dots, n-1\}, \quad F_n^{-1} = \frac{1}{n}\overline{F}
$$
$$
F_n = 
\begin{bmatrix} I_m & D_m \\ I_m & -D_m \end{bmatrix}
\begin{bmatrix} F_m & 0 \\ 0 & F_m \end{bmatrix}
\begin{bmatrix}
\text{even - odd} \\
\text{permutation}
\end{bmatrix}, \quad
(D_m)_{jj} = (w_m)^j, \quad j = 0,1,\dots , m-1, \quad m = n/2
$$

$$
y_j = y'_j + w_n^jy''_j, \quad y_{j+m} = y'_j - w_n^jy''_j, \quad j = 0,1,\dots,m-1
$$


\subsection*{Linear Algebra 4th Edition Strang Problem Set 10.3}
\begin{enumerate}
%%% 1 %%%
\item{(Problem 10.3.8)} Compute $y = F_8c$ by the three FFT steps for $c = (1,0,1,0,1,0,1,0)$. Repeat the 
computation for $c = (0,1,0,1,0,1,0,1)$. \\

We will introduce binary superscripts to denote even and odd splittings of $c$ for each recursive step. For $c = (1,0,1,0,1,0,1,0)$ we have

\begin{align*}
c^0 = (1,1,1,1),\; &\; c^1 = (0,0,0,0) \\
c^{00} = (1,1),\; c^{01} = (1,1)\;\; & \;\; c^{10} = (0,0),\; c^{11} = (0,0)\\
c^{000} = (1), \; c^{001} = (1) \quad \quad c^{010} = (1), \; c^{011} = (1) \qquad & \quad c^{100} = (0), \; c^{101} = (0) \quad \quad c^{110} = (0), \; c^{111} = (0) 
\end{align*} 

To calculate $y = F_8 c$ using the FFT, we need to use similar notation to track the recursive splittings of vectors based on $y$. \\

\begin{align*}
y_j = y_j^0 + w_8^j y_j^1, \enspace & \enspace
y_{j+4} = y_j^0 - w_8^j y_j^1, 
\enspace j = 0,1,2,3 \\
&\\
y_j^0 = y_j^{00} + w_4^j y_j^{01}, \enspace & \enspace
y_{j+2}^0 = y_j^{00} - w_4^j y_j^{01}, \enspace j = 0,1 \\
y_j^1 = y_j^{10} + w_4^j y_j^{11}, \enspace & \enspace
y_{j+2}^1 = y_j^{10} -w_4^j y_j^{11}, \enspace j = 0,1 \\
&\\
y_j^{00} = y_j^{000} + w_2^j y_j^{001}, \enspace & \enspace
y_{j+1}^{00} = y_j^{000} - w_2^j y_j^{001}, \enspace j = 0 \\
y_j^{01} = y_j^{010} + w_2^j y_j^{011}, \enspace & \enspace
y_{j+1}^{01} = y_j^{010} - w_2^j y_j^{011}, \enspace j = 0 \\
y_j^{10} = y_j^{100} + w_2^j y_j^{101}, \enspace & \enspace
y_{j+1}^{10} = y_j^{100} - w_2^j y_j^{101}, \enspace j = 0 \\
y_j^{11} = y_j^{110} + w_2^j y_j^{111}, \enspace & \enspace
y_{j+1}^{11} = y_j^{110} - w_2^j y_j^{111}, \enspace j = 0 \\
&\\
y^{000} = c^{000}, \enspace & y^{001} = c^{001}, \dots , y^{111} = c^{111}
\end{align*}

To get $y$, work backwards from the base case. Based on the given $c$, many of the terms in the expansion are zeroed out. 

\begin{align*}
y^{000} = 1, y^{001} = 1, y^{010} = 1, y^{011} = 1,& 
y^{100} = 0, y^{101} = 0, y^{110} = 0, y^{111} = 0 \\
&\\
y_0^{00} = 1+1 = 2, \enspace  \enspace
y_1^{00} = 1 -1 = 0,\implies y^{00} &= (2,0) \\
y_0^{01} = 1 +1 = 2, \enspace  \enspace
y_{1}^{01} = 1-1 = 0, \implies y^{01} &= (2,0) \\
y^{10} &= (0,0) \\
y^{11} &= (0,0) \\
&\\
y_0^0 = 4, \enspace 
y_1^0 = 0, \enspace &
y_{2}^0 = 0, \enspace 
y_{3}^0 = 0\\
\implies y^0 &= (4,0,0,0)\\
y^1 & = (0,0,0,0) \\
&\\
y &= (4,0,0,0,4,0,0,0)
\end{align*}

To check this result, calculate the matrix multiplication $F_8c$, which gives $y = (4,0,0,0,4,0,0,0)$. \\

For $c = (0,1,0,1,0,1,0,1)$, we have

$$
(c^{000}, c^{001}, c^{010}, c^{011}, c^{100}, c^{101},  c^{110},  c^{111})  = (0,0,0,0,1,1,1,1) 
$$

This means all terms with 0 leading in the superscript like $y^0, y^{00}, y^{01}, y^{000}, y^{010}, y^{001}, y^{011}$ are zero. Only terms with 1 leading in the superscript are nonzero. Less detail will be shown in this calculation since it is similar to the previous calculation. Only nonzero $y^b$ vectors written out. 

\begin{align*}
y^{10}_0 &= 1 + 1 = 2, y_1^{10} = 1 - 1 = 0 \implies y^{10} = (2,0) \\
y^{11}_0 &= 1 + 1 = 2, y_1^{11} = 1 - 1 = 0 \implies y^{11} = (2,0) \\
y_0^1 &= 2 + 2 = 4, y_1^1 = 0, y_2^1 = 2- 2= 0, y_3^1 = 0 \implies y^1 = (4,0,0,0) \\
&\\
y &= (4,0,0,0,-4,0,0,0)
\end{align*}

%%% 2 %%%
\item{(Problem 10.3.9)}

If $w = e^{2\pi i / 64}$ then $w^2$ and $\sqrt{w}$ are among the \rule{1cm}{0.15mm} and \rule{1cm}{0.15mm} roots of 1. \\

$w^2 = e^{2\pi i / 32}$ so $w^2$ is among the 32nd roots of 1. \\
$\sqrt{w} = e^{2\pi i / 128}$ so $\sqrt{w}$ is among the 128th roots of 1. 

%%% 3 %%%
\item{(Problem 10.3.13)}

\begin{enumerate}
	\item
	Two eigenvectors of $C$ are $(1, 1, 1, 1)$ and
	$(1, i, i^2 , i^3)$. Find the eigenvalues.
	
	$$
	\begin{bmatrix}
	c_0 & c_1 & c_2 & c_3 \\
	c_3 & c_0 & c_1 & c_2 \\
	c_2 & c_3 & c_0 & c_1 \\
	c_1 & c_2 & c_3 & c_0 
	\end{bmatrix}
	\begin{bmatrix}
	1 \\ 1 \\ 1 \\ 1
	\end{bmatrix}
	=e_1
	\begin{bmatrix}
	1 \\ 1 \\ 1 \\ 1
	\end{bmatrix}
	\quad \text{and} \quad
	\begin{bmatrix}
	c_0 & c_1 & c_2 & c_3 \\
	c_3 & c_0 & c_1 & c_2 \\
	c_2 & c_3 & c_0 & c_1 \\
	c_1 & c_2 & c_3 & c_0 
	\end{bmatrix}
	\begin{bmatrix}
	1 \\ i \\ i^2 \\ i^3
	\end{bmatrix}
	=e_2
	\begin{bmatrix}
	1 \\ i \\ i^2 \\ i^3
	\end{bmatrix}
	$$
	
	The matrix equations give the systems of equations:
	
	\begin{align*}
	c_0 + c_1 + c_2 + c_3 = e_1 \quad & \quad 
	c_0 + ic_1 + i^2c_2 + i^3c_3 = e_2\\
	c_3 + c_0 + c_1 + c_2 = e_1 \quad & \quad 
	c_3 + c_0i + c_1i^2 + c_2i^3 = ie_2 \\
	c_2 + c_3 + c_0 + c_1 = e_1 \quad & \quad 
	c_2 + c_3i + c_0i^2 + c_1i^3 = i^2e_2\\
	c_1 + c_2 + c_3 + c_0 = e_1 \quad & \quad 
	c_1 + c_2i + c_3i^2 + c_0i^3 = i^3e_2\\
	\end{align*}
	
	The four equations in the first system of equations all
	agree that $e_1 = c_0 + c_1 + c_2 + c_3$. All four equations
	in the second system of equations are all multiples of the
	first equation. That is, all equations agree that
	$e_2 = c_0 + ic_1 + i^2c_2 + i^3c_3$.
	
	\item
	$P = F\Lambda F^{-1}$ immediately gives 
	$P^2 = F\Lambda^2 F^{-1}$ and $P^3 = F\Lambda^3 F^{-1}$.
	Then
	\begin{align*}
	C &= c_0I + c_1P + c_2P^2 + c_3P^3 \\
	&= c_0FIF^{-1} + c_1F\Lambda F^{-1} +c_2F\Lambda^2 F^{-1}
	+c_3 F\Lambda^3 F^{-1} \\
	&= F(c_0I + c_1\Lambda + c_2 \Lambda^2 
	+ c_3 \Lambda^3)F^{-1} \\
	&= FEF^{-1} \;.
	\end{align*}
	
	The matrix $E$ is diagonal. It contains the eigenvalues of
	$C$. 

\end{enumerate}

%%% 4 %%%
\item{(Problem 10.3.14)}

Find the eigenvalues of the "periodic" $-1,2, -1$ matrix from 
$E = 2I - \Lambda - \Lambda^3$, with the eigenvalues of $P$ in $\Lambda$. 

$$
C=
\begin{bmatrix}
2 & -1 & 0 & -1 \\
-1 & 2 & -1 & 0 \\
0 & -1 & 2 & -1 \\
-1 & 0 & -1 & 2
\end{bmatrix}
\quad \text{ has } \quad
c_0 = 2, c_1 = -1, c_2 = 0, c_3 = -1 \;.
$$

Eigenvalues $e_l = 2-1-1 = 0$, $e_2 = 2-i -i^3 = 2$, $e_3 = 2- (-1) - (-1) = 4$, $e_4 = 2 - i^3 - i^9 = 2 +i -i = 2$. 

%%% 5 %%%
\item{(Problem 10.3.15)}

To multiply $C$ times a vector $x$, we can multiply $F (E (F^{-1} x ))$ 
instead. The direct way uses $n^2$ separate multiplications. Knowing $E$ and $F$, the 
second way uses only $n \log_2 n + n$ multiplications. How many of those come from 
$E$, how many from $F$, and how many from $F^{-1}$?\\

Diagonal E needs n multiplications, Fourier matrices $F$ and $F^{-1}$ need $\frac{1}{2}n \log_2 n$ multiplications each by the FFT.

\end{enumerate}

\subsection*{Introduction to Applied Mathematics Strang Problem Set 5.5}

\begin{enumerate}

%%% 1 %%%
\item{(Problem 5.5.1)}

What are $F^2$ and $F^4$ for the 4 by 4 Fourier matrix $F$. 

$$
F =
\begin{bmatrix}
1 & 1 & 1 & 1 \\
1 & i &  -1 & -i \\
1 & -1 & 1 & -1 \\
1 & -i & -1 & i
\end{bmatrix}
\;,\quad 
F^2 =
\begin{bmatrix}
4 & 0 & 0 & 0 \\
0 & 0 &  0 & 4 \\
0 & 0 & 4 & 0 \\
0 & 4 & 0 & 0
\end{bmatrix}
\;, \quad
F^4 = 16I_{4\times 4} \;.
$$

%%% 1 %%%
\item{(Problem 5.5.5)}

Compute $y = F_4x$ by the three steps of the FFT for the even vector $x' = (2,6,6,6)$ and the odd vector $x'' = (0,-2,0,2)$.

Let $x^0 = x'$, $y^0 = y$, and $F = F_4$.

\begin{align*}
x^{00} = (2,6)  \qquad \quad & \quad \qquad x^{01} = (6,6) \\
x^{000} = (2) \quad  x^{001} = (6) & \quad x^{010} = (6) \quad x^{011} = (6)
\end{align*}
\begin{align*}
& y^{000} = (2) \quad  y^{001} = (6) \quad y^{010} = (6) \quad y^{011} = (6) \\
& y^{00}_0 = 2 + 6 = 8 \quad y^{00}_1 = 2-6 = -4 \implies y^{00} = (8,-4) \\
& y^{01}_0 = 6 + 6 = 12 \quad y^{01}_1 = 6 - 6 = 0 \implies  y^{01} = (12,0) \\
&y^0_0 = 8 +12 = 20 \quad y^0_1 = -4 + 0i = -4 \quad y^0_2 = 8-12 = -4 \quad y^0_3 = -4-0i = -4 \\
& \implies y^0 = (20,-4,-4,-4)
\end{align*}

Let $x^1 = x''$, $y^1 = y$, and $F = F_4$.

\begin{align*}
x^{10} = (0,0)  \qquad \quad & \quad \qquad x^{11} = (-2,2) \\
x^{100} = (0) \quad  x^{101} = (0) & \quad x^{110} = (-2) \quad x^{111} = (2)
\end{align*}
\begin{align*}
& y^{100} = (0) \quad  y^{101} = (0) \quad y^{110} = (-2) \quad y^{111} = (2) \\
& y^{10}_0 = 0 + 0 = 0 \quad y^{10}_1 = 0-0 = 0 \implies y^{10} = (0,0) \\
& y^{11}_0 = -2 + 2 = 0 \quad y^{11}_1 = -2 - 2 = -4 \implies  y^{11} = (0,-4) \\
&y^1_0 = 0 +0 = 0 \quad y^1_1 = 0 + (-4)i = -4i \quad y^1_2 = 8-12 = 0 - 0 = 0 \quad y^1_3 = 0 - (-4)i = 4i \\
& \implies y^1 = (0,-4i,0,4i)
\end{align*}

%%% 1 %%%
\item{(Problem 5.5.6)}

What is $y = F_8x$ if $x = (2,0,6,-2,6,0,6,2)$? \\

Let $x^0 = (2,6,6,6)$ be the even splitting of $x$ and $x^{1} = (0,-2,0,2)$ be the odd splitting of $x$. From the previous problem we know that $F_4x^0 = y^0 = (20,-4,-4,-4)$ and $F_4x^1 = y^1 = (0,-4i, 0, 4i)$. Using the results from Problem 5.5.5 along with equation (2) on page 449,
\begin{align*}
y_j &= y^0_j + w_8^jy^1_j, \quad y_{j+m} =  y^0_j - w_8^jy^1_j, \quad j = 0,1,2,3, \quad w_8 = e^{2\pi i/ 8} = e^{i\pi/4} \;. \\
&\\
y_0 &= y^0_0 + w_8^0y^1_0 = 20 + 0 = 20 \\
y_1 &= y^0_1 + w_8^1y^1_1 = -4 + w_8^1(-4i) = -4 - 4iw_8\\
y_2 &= y^0_2 + w_8^2y^1_2  = -4 + w_8^2(0) = -4\\
y_3 &= y^0_3 + w_8^3y^1_3 = -4 + w_8^3(4i) = -4 + 4iw_8^3  \\
y_4 &= y^0_0 - w_8^0y^1_0 20 - 0 = 20\\
y_5 &= y^0_1 - w_8^1y^1_1 = -4 - w_8^1(-4i) = -4 + 4iw_8 \\
y_6 &= y^0_2 - w_8^2y^1_2 -4 - w_8^2(0) = -4\\
y_7 &= y^0_3 - w_8^3y^1_3 -4 - w_8^3(4i) = -4 - 4iw_8^3\\
\therefore y &= (20, -4-4iw_8, -4, -4+4iw_8^3, 20, -4+4iw_8, -4, -4-4iw_8^3)
\end{align*}

\end{enumerate}


\begin{enumerate}
\newpage
\section*{Assigned Problems}
%%% 1 %%%
\item

See Section 1 of HW4Script.m

\begin{enumerate}
	\item
	Let $f(x) = -x^2 + 2\pi x$. Sample $f$ at the $n=4$ points
	$x_j = 2\pi j / 4$ for $j = 0,1,2,3$. Find coefficients
	$c_0, c_1, c_2, c_3$ such that 
	$p(x) = c_0 +c_1e^{ix} + c_2e^{2ix} +c_3e^{3ix}$ matches $f(x)$ at 
	$x = 0, \pi /2 , \pi , 3\pi/2$.\\
	
	Let $f_j := f(x_j)$ and $w := e^{2\pi i / 4}$. The goal is to solve
	for $c$ in the following equation. 
	
	 $$
	 F_4c =
	 \begin{bmatrix}
	 1 & 1 & 1 & 1\\
	 1 & w & w^2 & w^3 \\
	 1 & w^2 & w^4 & w^6 \\
	 1 & w^3 & w^6 & w^9
	 \end{bmatrix}
	 \begin{bmatrix}
	 c_0 \\ c_1 \\ c_2 \\ c_3
	 \end{bmatrix}
	 = 
	 \begin{bmatrix}
	 f_0 \\ f_1 \\ f_2 \\ f_3
	 \end{bmatrix} = f
	 $$
	 
	 Since $F_4\overline{F}_4 = 4I$, 
	 $F_4^{-1} = \frac{1}{4}\overline{F}_4$. Therefore
	 $c = \frac{1}{4}\overline{F}_4 f$. To solve for the coefficients
	 component-wise, use the formula:
	 
	 $$
	 c_k = \frac{1}{4}\sum_{j=0}^3 f_j\overline{w}^{jk}
	 = \frac{1}{4} \sum_{j=0}^3 f_j e^{- i j k 2\pi / 4}
	 = \frac{1}{4} \sum_{j=0}^3 f_j e^{-ikx_j}  \;.
	 $$
	 
	 Using matrices:
	 
	 \begin{align*}
	 \begin{bmatrix}
	 c_0 \\ c_1 \\ c_2 \\ c_3
	 \end{bmatrix}
	 &= \frac{1}{4}
	 \begin{bmatrix}
	 1 & 1 & 1 & 1\\
	 1 & \overline{w} & \overline{w}^2 & \overline{w}^3 \\
	 1 & \overline{w}^2 & \overline{w}^4 & \overline{w}^6 \\
	 1 & \overline{w}^3 & \overline{w}^6 & \overline{w}^9
	 \end{bmatrix}
	 \begin{bmatrix}
	 f_0 \\ f_1 \\ f_2 \\ f_3
	 \end{bmatrix} \\
	 &= \frac{1}{4}
	 \begin{bmatrix}
	 1 & 1 & 1 & 1\\
	 1 & e^{-\pi i / 2} & e^{-\pi i} & e^{-3\pi i / 2} \\
	 1 & e^{-\pi i} & e^{-2\pi i} & e^{-3\pi i} \\
	 1 & e^{-3\pi i / 2} & e^{-3\pi i} & e^{- 9\pi i / 2}
	 \end{bmatrix}
	 \begin{bmatrix}
	 0 \\ \frac{3\pi^2}{4} \\ \pi^2 \\ \frac{3\pi^2}{4}
	 \end{bmatrix} \\
	 &= \frac{1}{4}
	 \begin{bmatrix}
	 1 & 1 & 1 & 1\\
	 1 & -i & -1 & i\\
	 1 & -1 & 1 & -1 \\
	 1 & i & -1 & -i
	 \end{bmatrix}
	  \begin{bmatrix}
	 0 \\ \frac{3\pi^2}{4} \\ \pi^2 \\ \frac{3\pi^2}{4}
	 \end{bmatrix} \\
	 &= 
	 \begin{bmatrix}
	 \frac{5\pi^2}{8}\\ -\frac{\pi^2}{4} \\ -\frac{\pi^2}{8} \\
	 -\frac{\pi^2}{4}
	 \end{bmatrix} \\
	 \end{align*}
	 
	 $$
	 \boxed{
	 p(x)= 
	 \frac{5\pi^2}{8} -\frac{\pi^2}{4}e^{ix} -\frac{\pi^2}{8}e^{2ix} 
	 -\frac{\pi^2}{4}e^{3ix}
	 }
	 $$
	 
	 \newpage
	 
	 
	 \item
	 Find coefficients $c_{-2}, c_{-1}, c_0, c_1$ such
	 that $p(x) =  c_{-2}e^{-2ix} + c_{-1}e^{-ix} + c_0 + c_1e^{ix}$
	 matches $f(x)$ at $x = 0, \pi/2, \pi, 3\pi /2$. This amounts to 
	 solving for $c$ in the following matrix equation. 
	 
	 \begin{align*}
	 F_4c = 
	 \begin{bmatrix}
	 1 & 1 & 1 & 1\\
	 w^{-2} & w^{-1} & 1 & w^1 \\
	 w^{-4} & w^{-2} & 1 & w^2 \\
	 w^{-6} & w^{-3} & 1 & w^3
	 \end{bmatrix}
	 \begin{bmatrix}
	 c_{-2}\\ c_{-1} \\ c_0 \\ c_1
	 \end{bmatrix}
	 = 
	 \begin{bmatrix}
	 f_0 \\ f_1 \\ f_2 \\ f_3
	 \end{bmatrix} = f
	 \end{align*}
	 
	 While $f$ is the same here as in part a, $F_4$ and $c$ are not the
	 same as in part a. However, the columns of this version of $F_4$
	 are still orthogonal with length 4. Again we have 
	$F_4 \overline{F}_4 = 4I$ so that $c = \frac{1}{4}\overline{F}_4f$.
	
	\begin{align*}
	\begin{bmatrix}
	 c_{-2}\\ c_{-1} \\ c_0 \\ c_1
	 \end{bmatrix}
	 &= \frac{1}{4}
	 \begin{bmatrix}
	 1 & 1 & 1 & 1\\
	 \overline{w}^{-2} & \overline{w}^{-1} & 1 & \overline{w}^1 \\
	 \overline{w}^{-4} & \overline{w}^{-2} & 1 & \overline{w}^2 \\
	 \overline{w}^{-6} & \overline{w}^{-3} & 1 & \overline{w}^3
	 \end{bmatrix}
	 \begin{bmatrix}
	 f_0 \\ f_1 \\ f_2 \\ f_3
	 \end{bmatrix} \\
	 &= \frac{1}{4}
	 \begin{bmatrix}
	 1 & 1 & 1 & 1\\
	 e^{- \pi i } & e^{- \pi i /2} & 1 & e^{\pi i /2} \\
	 e^{- 2\pi i } & e^{- \pi i } & 1 & e^{\pi i } \\
	 e^{- 3\pi i } & e^{- 3\pi i /2} & 1 & e^{3\pi i / 2 }
	 \end{bmatrix}
	 \begin{bmatrix}
	 0 \\ \frac{3\pi^2}{4} \\ \pi^2 \\ \frac{3\pi^2}{4}
	 \end{bmatrix} \\
	  &= \frac{1}{4}
	 \begin{bmatrix}
	 1 & 1 & 1 & 1\\
	 -1 & -i & 1 & i \\
	 1 & -1 & 1 & -1 \\
	 -1 & i & 1 & -i
	 \end{bmatrix}
	 \begin{bmatrix}
	 0 \\ \frac{3\pi^2}{4} \\ \pi^2 \\ \frac{3\pi^2}{4}
	 \end{bmatrix} \\
	 &= \begin{bmatrix}
	 c_{-2}\\ c_{-1} \\ c_0 \\ c_1
	 \end{bmatrix}
	\end{align*} 
	
	$$
	 \boxed{
	 p(x)= 
	 \frac{5\pi^2}{8}e^{-2ix} +\frac{\pi^2}{4}e^{-ix} -\frac{\pi^2}{8}
	 +\frac{\pi^2}{4}e^{ix}
	 }
	 $$ 
	 
\end{enumerate}

\item

See Section 2 of HW4Script.m 


\end{enumerate}


\end{document}